To investigate the impact of the proposed waveform on the harvested power, we apply the received signal expression \ref{eqn:received_signal_component} to the diode current equation \ref{eqn:output_current}.

First, we consider the multicarrier multisine waveform ${y_P}(t)$. The approximated harvester DC current with multisine excitation writes as

\begin{equation}\label{eqn:current_power}
  {i_{\text{out}}} \approx k_0^\prime  + \sum\limits_{i{\text{ even }},i \geqslant 2}^{{n_o}} {k_i^\prime } {\rho ^{i/2}}R_{\text{ant}}^{i/2}\mathbb{E}\left[ {{y_P}{{(t)}^i}} \right]
\end{equation}

Also, it is derived in \cite{Clerckx2016} that the expectation of the received power waveform to the second and fourth orders write respectively as

\begin{align}\label{eqn:power_waveform_second_order}
  \mathbb{E}\left[ {{y_P}{{(t)}^2}} \right] &= \frac{1}{2}\sum\limits_{n = 0}^{N - 1} {{{\left| {{{\mathbf{h}}_n}{{\mathbf{w}}_{P,n}}} \right|}^2}} \\
   &= \frac{1}{2}\sum\limits_{n = 0}^{N - 1} {\sum\limits_{{m_0},{m_1}} {{s_{P,n,{m_0}}}{s_{P,n,{m_1}}}{A_{n,{m_0}}}{A_{n,{m_1}}}\cos \left( {{\psi _{P,n,{m_0}}} - {\psi _{P,n,{m_1}}}} \right)} }
\end{align}

\begin{align}\label{eqn:power_waveform_fourth_order}
  \mathbb{E}\left[ {{y_P}{{(t)}^4}} \right] &= \frac{3}{8}\Re \left\{ {\sum\limits_{\substack{{n_0},{n_1},{n_2},{n_3} \\ {n_0} + {n_1} = {n_2} + {n_3}}} {{{\mathbf{h}}_{{n_0}}}{{\mathbf{w}}_{P,{n_0}}}{{\mathbf{h}}_{{n_1}}}{{\mathbf{w}}_{P,{n_1}}}{{\left( {{{\mathbf{h}}_{{n_2}}}{{\mathbf{w}}_{P,{n_2}}}} \right)}^*}{{\left( {{{\mathbf{h}}_{{n_3}}}{{\mathbf{w}}_{P,{n_3}}}} \right)}^*}} } \right\} \\
   &= \frac{3}{8}\sum\limits_{\substack{{n_0},{n_1},{n_2},{n_3} \\ {n_0} + {n_1} = {n_2} + {n_3}}} {\sum\limits_{{m_0},{m_1},{m_2},{m_3}} {\left[ {\prod\limits_{j = 0}^3 {{s_{{P},{n_j},{m_j}}}} {A_{{n_j},{m_j}}}} \right] }} \nonumber \\
   &\quad \cos \left( {{\psi _{{P},{n_0},{m_0}}} + {\psi _{{P},{n_1},{m_1}}} - {\psi _{{P},{n_2},{m_2}}} - {\psi _{{P},{n_3},{m_3}}}} \right)
\end{align}

We then turn to the multicarrier modulated waveform ${y_I}(t)$. It can be treated as a multisine waveform for the input symbols $\{ {{\tilde x}_n}\} $ that vary randomly with symbol rate $1/{B_{\text{s}}}$. Similarly, the approximated DC current provided by the rectifier is given by

\begin{equation}\label{eqn:current_information}
  {i_{\text{out}}} \approx k_0^\prime  + \sum\limits_{i{\text{ even }},i \geqslant 2}^{{n_o}} {k_i^\prime } {\rho ^{i/2}}R_{\text{ant}}^{i/2}{\mathbb{E}_{\{ {{\tilde x}_n}\} }}\left[ {{y_I}{{(t)}^i}} \right]
\end{equation}

To obtain the expectation, we first extract the DC currents corresponding to a given set of amplitudes $\{ {{\tilde s}_{I,n,m}}\} $ and phases $\{ {{\tilde \phi }_{I,n,m}}\} $, then take the expectation over the distribution of the input symbol ${{\tilde x}_n}$. As an i.i.d. CSCG distribution ${{\tilde x}_n}\sim\mathcal{C}\mathcal{N}(0,1)$ is assumed, the amplitude square ${\left| {{{\tilde x}_n}} \right|^2}$ is exponentially distributed with $\mathbb{E}\left[ {{{\left| {{{\tilde x}_n}} \right|}^2}} \right] = 1$. Using the moment generating function, we also have $\mathbb{E}\left[ {{{\left| {{{\tilde x}_n}} \right|}^4}} \right] = \mathbb{E}\left[ {{{\left( {{{\left| {{{\tilde x}_n}} \right|}^2}} \right)}^2}} \right] = 2$. Note this gain applies to the output current, which measures the contribution of modulation and does not exist for multisine waveform. Following \cite{Clerckx2018}, we can obtain the expectation of the received information waveform to the second and fourth orders

\begin{align}\label{eqn:information_waveform_second_order}
  \mathbb{E}\left[ {{y_I}{{(t)}^2}} \right] &= \frac{1}{2}\sum\limits_{n = 0}^{N - 1} {\sum\limits_{{m_0},{m_1}} {{s_{I,n,{m_0}}}} } {s_{I,n,{m_1}}}{A_{n,{m_0}}}{A_{n,{m_1}}}\cos \left( {{\psi _{I,n,{m_0}}} - {\psi _{I,n,{m_1}}}} \right) \\
   &= \frac{1}{2}\sum\limits_{n = 0}^{N - 1} {{{\left| {{{\mathbf{h}}_n}{{\mathbf{w}}_{I,n}}} \right|}^2}}
\end{align}

\begin{align}\label{eqn:information_waveform_fourth_order}
  \mathbb{E}\left[ {{y_I}{{(t)}^4}} \right] &= \frac{6}{8}\sum\limits_{{n_0},{n_1}} {\sum\limits_{{m_0},{m_1},{m_2},{m_3}} {\left[ {\prod\limits_{j = 0,2} {{s_{I,{n_0},{m_j}}}{A_{{n_0},{m_j}}}} } \right]\left[ {\prod\limits_{j = 1,3} {{s_{I,{n_1},{m_j}}}{A_{{n_1},{m_j}}}} } \right]} } \nonumber \\
   &\quad \cos \left( {{\psi _{I,{n_0},{m_0}}} + {\psi _{I,{n_1},{m_1}}} - {\psi _{I,{n_0},{m_2}}} - {\psi _{I,{n_1},{m_3}}}} \right) \\
   &= \frac{6}{8}{\left[ {\sum\limits_{n = 0}^{N - 1} {{{\left| {{{\mathbf{h}}_n}{{\mathbf{w}}_{I,n}}} \right|}^2}} } \right]^2} \label{eqn:waveform_end}
\end{align}

It is worth noting that the truncation order ${n_o}$ in equation \ref{eqn:current_information} determines the relationship between the received signal and the harvested power. On top of it, \cite{Clerckx2016} proposed two diode models:

\begin{itemize}
  \item \textit{diode linear model} (${n_o} = 2$) is the conventional perspective that assumes the total output power is the sum of the subband power. It omits the rectifier nonlinearity and is typically suitable for a very low input power (below -30 dBm).
  \item \textit{diode nonlinear model} (${n_o} > 2$) considers the contributions of higher order terms to the harvested power. It captures the nonlinear behavior of the diode with the product terms that consist of contributions from different frequencies (as indicated by ${{n_0},{n_1}}$ in equation \ref{eqn:information_waveform_fourth_order} and \ref{eqn:power_waveform_fourth_order}). The model is complicated but accurate, and especially fits the low power regime between -30 dBm and 0 dBm.
\end{itemize}

In the diode linear model corresponding to equations \ref{eqn:power_waveform_second_order} and \ref{eqn:information_waveform_second_order}, the output current is only a function of $\sum\limits_{n = 0}^{N - 1} {{{\left| {{{\mathbf{h}}_n}{{\mathbf{w}}_{P/I,n}}} \right|}^2}} $. Hence, it appears that multicarrier multisine and modulated waveforms are equally suitable for WPT. On the other hand, the diode nonlinear model highlights a clear difference between the power delivered by both waveforms. For the modulated component, the second and fourth order terms in \ref{eqn:information_waveform_second_order} and \ref{eqn:information_waveform_fourth_order} share same dependencies on ${\sum\limits_{n = 0}^{N - 1} {{{\left| {{{\mathbf{h}}_n}{{\mathbf{w}}_{I,n}}} \right|}^2}} }$. It implies that for a modulated waveform with CSCG inputs, the higher order terms behave similarly to the second order term and both models are equivalent. In comparison, for the multisine waveform, the terms \ref{eqn:power_waveform_second_order} and \ref{eqn:power_waveform_fourth_order} are decomposed as the product of contributions from different subbands. Also, the second order term is linear as a sum over each frequencies while the nonlinear fourth order term shows some cross correlation between different subbands.

In this paper, we set ${n_o} = 4$ to explore the fundamental nonlinear behaviour of the diode and its impact on the harvested current. Since $\mathbb{E}\left[ {{y_P}(t){y_I}(t)} \right] = \mathbb{E}\left[ {{y_P}{{(t)}^3}{y_I}(t)} \right] = \mathbb{E}\left[ {{y_P}(t){y_I}{{(t)}^3}} \right] = 0$ and $\mathbb{E}\left[ {{y_P}{{(t)}^2}{y_I}{{(t)}^2}} \right] = \mathbb{E}\left[ {{y_P}{{(t)}^2}} \right]\mathbb{E}\left[ {{y_I}{{(t)}^2}} \right]$, the approximated output DC current of equation \ref{eqn:output_current} reduces to

\begin{align}\label{eqn:output_current_truncated}
  {i_{\text{out}}} &\approx k_0^\prime  + k_2^\prime \rho {R_{{\text{ant}}}}\mathbb{E}\left[ {{y_P}{{(t)}^2}} \right] + k_4^\prime {\rho ^2}R_{ant}^2\mathbb{E}\left[ {{y_P}{{(t)}^4}} \right] \nonumber \\
   &\quad + k_2^\prime \rho {R_{{\text{ant}}}}\mathbb{E}\left[ {{y_I}{{(t)}^2}} \right] + k_4^\prime {\rho ^2}R_{ant}^2\mathbb{E}\left[ {{y_I}{{(t)}^4}} \right] \nonumber \\
   &\quad + 6k_4^\prime {\rho ^2}R_{{\text{ant}}}^2\mathbb{E}\left[ {{y_P}{{(t)}^2}} \right]\mathbb{E}\left[ {{y_I}{{(t)}^2}} \right]
\end{align}

whose corresponding target function is

\begin{align}\label{eqn:target_function_truncated}
  {z_{\text{DC}}} &\approx k_0  + k_2 \rho {R_{{\text{ant}}}}\mathbb{E}\left[ {{y_P}{{(t)}^2}} \right] + k_4 {\rho ^2}R_{ant}^2\mathbb{E}\left[ {{y_P}{{(t)}^4}} \right] \nonumber \\
   &\quad + k_2 \rho {R_{{\text{ant}}}}\mathbb{E}\left[ {{y_I}{{(t)}^2}} \right] + k_4 {\rho ^2}R_{ant}^2\mathbb{E}\left[ {{y_I}{{(t)}^4}} \right] \nonumber \\
   &\quad + 6k_4 {\rho ^2}R_{{\text{ant}}}^2\mathbb{E}\left[ {{y_P}{{(t)}^2}} \right]\mathbb{E}\left[ {{y_I}{{(t)}^2}} \right]
\end{align} 