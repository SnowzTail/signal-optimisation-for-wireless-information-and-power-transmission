Consider a point-to-point MISO WIPT system in multipath environment. The $M$-antenna transmitter delivers information and power simultaneously to the single-antenna receiver through $N$ orthogonal subbands. It is assumed the carrier frequencies are with even spacing $\Delta f$ and equal bandwidth ${B_{\text{s}}}$. The $n$-th subband has carrier frequency ${f_n} = {f_0} + n\Delta f$ for $n = 0, \ldots ,N - 1$. To maximize the rate-energy tradeoff, we employ a superposed signal consists of a multi-carrier deterministic multisine waveform and a multi-carrier random modulated waveform for WIPT. Both components are transmitted on the same frequency bands.



\subsection{Transmitted Information Waveform}\label{sec:transmitted-information-waveform}
Denote the information symbol carried by the modulated waveform on subband $n$ as ${{\tilde x}_n}$, we assume the input symbol is with the capacity-achieving i.i.d. Circular Symmetric Complex Gaussian (CSCG) distribution with zero mean and unit variance \cite{Varasteh2017a}:

\begin{equation}\label{eqn:unmodulated_symbol}
  {{\tilde x}_n} = \left| {{{\tilde x}_n}} \right|{e^{j{\phi _{{{\tilde x}_n}}}}}\sim\mathcal{C}\mathcal{N}(0,1)
\end{equation}

Hence, the modulated waveform on antenna $m = 1, \ldots ,M$, subband $n$ writes as

\begin{equation}\label{eqn:modulated_symbol}
  {x_{n,m}} = {w_{I,n,m}}{{\tilde x}_n}
\end{equation}

where ${w_{I,n,m}}$ is the corresponding information weight and is a constant for a certain channel state:

\begin{equation}\label{eqn:weight_information}
  {w_{I,n,m}} = \left| {{w_{I,n,m}}} \right|{e^{j{\phi _{I,n,m}}}} = {s_{I,n,m}}{e^{j{\phi _{I,n,m}}}}
\end{equation}

Note the amplitude and phase are separated in resource allocation. Define matrices ${{\mathbf{S}}_I}$ and ${{\mathbf{\Phi }}_I}$ of size $N \times M$ such that the $(n,m)$ entries hold ${s_{I,n,m}}$ and ${\phi _{I,n,m}}$ respectively, the design of information waveform is converted into an optimization problem on both matrices, with the average WIT transmit power ${P_I} = \frac{1}{2}\left\| {{{\mathbf{S}}_I}} \right\|_F^2$. The modulated symbol of equation \ref{eqn:modulated_symbol} can be further expressed as

\begin{equation}\label{eqn:modulated_symbol_further}
  {x_{n,m}} = {s_{I,n,m}}{e^{j{\phi _{I,n,m}}}} \cdot \left| {{{\tilde x}_n}} \right|{e^{j{\phi _{{{\tilde x}_n}}}}} = {{\tilde s}_{I,n,m}}{e^{j{{\tilde \phi }_{I,n,m}}}}
\end{equation}

with ${{\tilde s}_{I,n,m}} = {s_{I,n,m}}\left| {{{\tilde x}_n}} \right|$ and ${{\tilde \phi }_{I,n,m}} = {\phi _{I,n,m}} + {\phi _{{{\tilde x}_n}}}$. In this way, the impact of symbol distribution and waveform design are combined. The modulated waveform also follows an i.i.d. CSCG distribution with variance equal to the subband power ${x_{n,m}}\sim\mathcal{C}\mathcal{N}\left( {0,s_{I,n,m}^2} \right)$.

Therefore, the information waveform ${x_{I,m}}(t)$ on antenna $m$ at time $t$ writes as

\begin{align}\label{eqn:information_waveform}
  {x_{I,m}}(t) &= \sum\limits_{n = 0}^{N - 1} {{{\tilde s}_{I,n,m}}(t)\cos \left( {2\pi {f_n}t + {{\tilde \phi }_{I,n,m}}(t)} \right)}  \hfill \\
   &= \Re \left\{ {\sum\limits_{n = 0}^{N - 1} {{x_{n,m}}(t){e^{j2\pi {f_n}t}}} } \right\} \hfill \\
   &= \Re \left\{ {\sum\limits_{n = 0}^{N - 1} {{w_{I,n,m}}{{\tilde x}_n}(t){e^{j2\pi {f_n}t}}} } \right\} \hfill
\end{align}

On top of this, the WIT signal vector is spread over $M$ antennas

\begin{equation}\label{eqn:wit_vector}
  {{\mathbf{x}}_I}(t) = \Re \left\{ {\sum\limits_{n = 0}^{N - 1} {{{\mathbf{w}}_{I,n}}} {{\tilde x}_n}(t){e^{j2\pi {f_n}t}}} \right\}
\end{equation}

where ${{\mathbf{w}}_{I,n}} = {\left[ {{w_{I,n,1}} \cdots {w_{I,n,M}}} \right]^T}$.



\subsection{Transmitted Power Waveform}\label{sec:transmitted-power-waveform}
Comparing with the information component, the multisine power component is unmodulated and deterministic, so there is no dependency on the distribution of input symbol $\tilde{x}_{n}(t)$. The power waveform on antenna $m$, subband $n$ is given by

\begin{equation}\label{eqn:unmodulated}
  {w_{P,n,m}} = {s_{P,n,m}}{e^{j{\phi _{P,n,m}}}}
\end{equation}

where ${s_{P,n,m}}$ and ${{\phi _{P,n,m}}}$ are the amplitude and phase of the multisine signal. Collect them into the $(n,m)$ entries of matrices ${{\mathbf{S}}_P}$ and ${{\mathbf{\Phi }}_P}$, the average power of the WPT waveform is $\frac{1}{2}\left\|\mathbf{S}_{P}\right\|_{F}^{2}$. Similarly, the power waveform ${x_{P,m}}(t)$ on antenna $m$ at time $t$ is

\begin{align}\label{eqn:power_waveform}
  {x_{P,m}}(t) &= \sum\limits_{n = 0}^{N - 1} {{s_{P,n,m}}\cos \left( {2\pi {f_n}t + {\phi _{P,n,m}}} \right)}  \hfill \\
   &= \Re \left\{ {\sum\limits_{n = 0}^{N - 1} {{w_{P,n,m}}{e^{j2\pi {f_n}t}}} } \right\} \hfill
\end{align}

Combine the power signals on all $M$ antennas, the WPT signal vector writes as

\begin{equation}\label{eqn:wpt_vector}
  {{\mathbf{x}}_P}(t) = \Re \left\{ {\sum\limits_{n = 0}^{N - 1} {{{\mathbf{w}}_{P,n}}} {e^{j2\pi {f_n}t}}} \right\}
\end{equation}

with ${{\mathbf{w}}_{P,n}} = {\left[ {{w_{P,n,1}} \cdots {w_{P,n,M}}} \right]^T}$.



\subsection{Multipath Channel and Received Signal}\label{sec:multipath-and-received-signal}
Consider a multipath channel with $L$ paths. For the $l$-th path ($l = 1, \ldots ,L$), denote the phase shift between the receive antenna and transmit antenna $m$ of subband $n$ as ${\zeta _{n,m,l}}$. Let ${\tau _l}$ and ${\alpha _l}$ be the delay and magnitude gain, and indicate the transmit signal on subband $n$ of antenna $m$ as

\begin{equation}\label{eqn:superposed_waveform}
  {v_{n,m}}(t) = {w_{P,n,m}} + {w_{I,n,m}}{{\tilde x}_n}(t)
\end{equation}

The superposed signal containing modulated information waveform and multisine power waveform is demonstrated to bring a two-fold benefit on rate and energy \cite{Clerckx2019}. Also, the channel frequency response is expressed as

\begin{equation}\label{eqn:channel}
  {h_{n,m}} = \sum\limits_{l = 0}^{L - 1} {{\alpha _l}{e^{j\left( { - 2\pi {f_n}{\tau _l} + {\zeta _{n,m,l}}} \right)}}}  = {A_{n,m}}{e^{j{{\bar \psi }_{n,m}}}}
\end{equation}

To ensure $v_{n, m}(t)$ and $\tilde{x}_{n}(t)$ being narrowband signals, we assume ${\max _{l \ne {l^\prime }}}\left| {{\tau _l} - {\tau _{{l^\prime }}}} \right| <  < 1/{B_{\text{s}}}$. It is also supposed that ${v_{n,m}}\left( {t - {\tau _l}} \right) = {v_{n,m}}(t)$ and ${{\tilde x}_n}\left( {t - {\tau _l}} \right) = {{\tilde x}_n}(t)$. The received signal corresponding to transmit antenna $m$ contains the power component $y_{P, m}(t)$ and the information component $y_{I, m}(t)$

\begin{align}\label{eqn:received_signal_component}
  {y_m}(t) &= {y_{P,m}}(t) + {y_{I,m}}(t) \hfill \\
   &= \Re \left\{ {\sum\limits_{l = 0}^{L - 1} {\sum\limits_{n = 0}^{N - 1} {{\alpha _l}} } {v_{n,m}}\left( {t - {\tau _l}} \right){e^{j2\pi {f_n}\left( {t - {\tau _l}} \right) + {\zeta _{n,m,l}}}}} \right\} \hfill \\
   &\approx \Re \left\{ {\sum\limits_{n = 0}^{N - 1} {{h_{n,m}}} {v_{n,m}}(t){e^{j2\pi {f_n}t}}} \right\} \hfill
\end{align}

Hence, the total received signal can be obtained by stacking up equation \ref{eqn:received_signal_component} over all transmit signals

\begin{align}\label{eqn:received_signal}
  y(t) &= {y_P}(t) + {y_I}(t) \hfill \\
   &= \Re \left\{ {\sum\limits_{n = 0}^{N - 1} {{{\mathbf{h}}_n}} \left( {{{\mathbf{w}}_{P,n}} + {{\mathbf{w}}_{I,n}}{{\tilde x}_n}} \right){e^{j2\pi {f_n}t}}} \right\} \hfill
\end{align}

where the channel vector is defined as ${{\mathbf{h}}_n} = \left[ {{h_{n,1}} \ldots {h_{n,M}}} \right]$. 