\chapter{Conclusions and Future Works}
This paper performed waveform optimization and rate-energy region characterization for a point-to-point WIPT. Based on the diode nonlinear model, a superposition of multi-carrier modulated and unmodulated waveforms at the transmitter are jointly optimized with the power splitter at the receiver. The signal design is modeled as a non-convex posynomial maximization problem adaptive to the CSI. We also extend the existing work to MIMO systems and consider the PAPR constraints.

Numerical results demonstrate the following conclusions. First, the harvester nonlinearity can be exploited to boost the harvested energy. It prefers a different waveform design, transceiver architecture, and resource allocation. Second, modulation benefits the delivered power in single-carrier transmission but is detrimental for multi-carrier WIPT. Third, the superposed signal can effectively enlarge the R-E region with the twofold benefit of multisine. Fourth, a combination of PS and TS is generally the optimal receiver strategy. Fifth, frequency selectivity has a positive influence on the harvested energy. Sixth, increasing Tx and/or Rx not only improves the rate-energy tradeoff but also reduces the PAPR requirement.

Some limitations of this work require further attention. First, the adaptive design relies on perfect CSIT and synchronization between transmitter and receiver, but both conditions can be hard to achieve. Second, the power splitting ratio can be difficult to adjust dynamically in practice. Third, the GP approach is suboptimal for MIMO systems. Fourth, the iterative algorithms are sensitive to initialization and time-consuming when a large number of subbands and/or antennas are employed.

Several novel ideas in recent research may be further integrated with this paper. For instance, the optimal approach for MIMO WIPT is discussed in \cite{Huang2017}. Also, \cite{Park2018} proposed a dual-mode SWIPT with an adaptive \textit{Mode Switching} (MS) algorithm to alternate between single-tone and multi-tone transmission. The former employs a multi-energy level signaling with PSK for high rate communication, while the latter modulates the multisine waveform by PAPR for power-demanding applications \cite{Krikidis2019}. A generic receiver architecture for MIMO WPT was designed in \cite{Ma2019}, which demonstrated that using multiple rectifiers with proper beamforming and power allocation scheme can significantly improve the harvested power. Moreover, another nonlinear EH model was proposed in \cite{Boshkovska2015} whose parameters relies on a logistic curve fitting technique. 