In this article, we depart from the rectifier behavior and diode characteristics to revisit the analytical harvester models proposed in \cite{Clerckx2016}. On top of this, a multi-carrier modulated waveform and a multi-carrier unmodulated waveform are compared in terms of harvested energy. The results demonstrated that when considering rectifier nonlinearity, the unmodulated waveform outperforms the modulated waveform for multi-carrier WPT but is outperformed for single-carrier transmission. It is in sharp contrast to the conventional opinion based on linear rectifier model.

We also explore the adaptive transceiver design in \cite{Clerckx2018} that jointly optimizes the superposed signal (consists of modulated and unmodulated components) at the transmitter and the power splitter at the receiver. The characterization of the R-E region is converted into an optimization problem maximizing the harvested current with discrete rate constraints, which relies on Geometric Programming (GP) tools. The original strategy for Multiple-Input Single-Output (MISO) is extended to Multiple-Input Multiple-Output (MIMO), and the Peak-to-Average Power Ratio (PAPR) constraint is introduced to the system. 