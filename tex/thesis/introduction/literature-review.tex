Significant improvements on WIPT has been witnessed in the last decade. The idea was first proposed in \cite{R.Varshney2008}, where the author first defined a nonlinear concave capacity-energy function and investigated the tradeoff for typical binary channels and a flat additive white Gaussian noise (AWGN) channel with amplitude-constrained inputs. It was extended to frequency-selective channel in \cite{Grover2010}. However, both works were based on the impractical assumption that information decoding (ID) and Energy Harvesting (EH) can be performed individually on the same received signal. In \cite{Zhang2013}, the authors proposed two practical co-located receiver designs named \textit{time switching} (TS) that switches between ID and EH and \textit{power splitting} (PS) that splits the received power into two separate streams. It was then demonstrated in \cite{Zhou2013a} that TS can guarantee the same rate as conventional Time-Division Multiple Access (TDMA) while providing considerable power. In comparison, PS may lead to higher rate when the demand on power is sufficiently high. A further research \cite{Liu2013} enabled dynamic power splitting that adjusts the power split ratio based on the channel state information (CSI), and proposed a suboptimal low-complexity \textit{antenna switching} scheme. Nevertheless, the literatures above are mostly based on an oversimplified linear harvester model. To accurately characterize the behavior of the rectenna, \cite{Clerckx2016} derived a tractable nonlinear model and performed an adaptive multisine waveform design accordingly. Realistic simulations showed significant gains in harvested power and stressed the importance of modelling rectifier nonlinearity in wireless system design. The work was extended to multi-input single-output (MISO) WIPT in \cite{Clerckx2018} where a superposition of multicarrier modulated and unmodulated waveform was optimized as a function of CSI under transmit power budget. It suggested the rectifier nonlinearity can lead to a larger rate-energy (R-E) region and favours a different waveform, modulation and input distribution. In another perspective, a learning approach \cite{Varasteh2018} modelled the transmitter and receiver as deep neural networks (NN) and jointly optimized signal encoding with network parameters. Constellations showed that the offset of the power symbol is positively correlated to the power demand, while the information symbols are symmetrically located around the origin. The pattern confirmed a unmodulated waveform is beneficial to increase the harvested power.
