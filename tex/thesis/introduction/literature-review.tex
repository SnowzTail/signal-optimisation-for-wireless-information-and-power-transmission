Significant achievements in WIPT has been witnessed in the last decade. The idea was first proposed in \cite{R.Varshney2008}, where the author defined a concave capacity-energy function and investigated the tradeoff for typical binary channels and a flat additive white Gaussian noise (AWGN) channel with amplitude-constrained inputs. The research was extended to frequency-selective channel in \cite{Grover2010}. However, both works were based on the impractical assumption that information decoding (ID) and Energy Harvesting (EH) can be performed individually on the same received signal. In \cite{Zhang2013}, the authors proposed two practical co-located receiver designs named \textit{time switching} (TS) that switches between ID and EH and \textit{power splitting} (PS) that splits the received power into two separate streams. It was then demonstrated in \cite{Zhou2013a} that TS can guarantee the same rate as conventional Time-Division Multiple Access (TDMA) while providing considerable power to the system. In comparison, PS may produce a higher rate when the power demand is sufficiently high. A further research \cite{Liu2013} proposed a suboptimal low-complexity \textit{antenna switching} scheme and enabled dynamic power splitting to adjust the split ratio with the channel state information (CSI). Nevertheless, the literature above relies on an oversimplified linear harvester model. To accurately characterize the behavior of the rectenna, \cite{Clerckx2016} derived a tractable nonlinear model and performed an adaptive multisine waveform design for WPT. Realistic simulations showed significant gains in harvested power and stressed the importance of modeling rectifier nonlinearity in wireless system design. The work was extended to multi-input single-output (MISO) WIPT in \cite{Clerckx2018}, where the multisine is superposed to a modulated information waveform and optimized adaptively to CSI. It suggested the rectifier nonlinearity can lead to a larger rate-energy (R-E) region, which favors a different waveform, modulation, and input distribution. In another perspective, a learning approach \cite{Varasteh2018} modeled the transmitter and receiver as deep neural network (NN) and jointly optimized signal encoding with network parameters. Constellations showed that the offset of the power symbol is positively correlated to the power demand, while the information symbols are symmetrically located around the origin. The pattern confirmed that an unmodulated waveform is beneficial to increase the harvested power.
