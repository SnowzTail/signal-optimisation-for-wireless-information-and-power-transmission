Energy-constrained wireless devices are conventionally powered by batteries. However, the development of large-scale networks as Internet-of-Things (IoT) is restricted by its limited working time and frequent recharging or replacement. Although Wireless Power Transfer (WPT) via inductive coupling has enjoyed some success in real-world applications, it is impractical for most devices on the move since the operation range is relatively short. As a promising alternative, the Radio-Frequency (RF) wave is with lower power level (\si{\uW} to \si{W}) but broader coverage (up to hundreds of meters) \cite{Ng2019}. Interestingly, it indeed carries both information and energy simultaneously, with the potential to power billions of mobile nodes wirelessly while keeping them connected. The recent revolution in harvester model and the significant drop of power requirement in electronics will bring more opportunities to the research on Wireless Information and Power Transfer (WIPT) via RF signals.
