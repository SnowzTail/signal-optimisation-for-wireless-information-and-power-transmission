\begin{abstract}
  Wireless can be more than communications. As a medium of information and energy, the radio wave enables a unified Wireless Information and Power Transfer (WIPT) to link and charge low-power devices remotely. This paper departs from the rectifier behavior to derive a nonlinear energy harvester model and investigate the relationship between waveform design and power transmission. On top of that, a superposition of multi-carrier modulated and unmodulated multisine waveforms is introduced to improve the rate-energy (R-E) tradeoff. With an adaptive transceiver design, we jointly optimize the superposed signal at the transmitter and the power splitter at the receiver according to the channel state information. We then extend the existing works to MIMO and consider the influence of frequency selectivity and PAPR constraints. Based on non-convex posynomial maximization, the iterative algorithms are demonstrated to enlarge the R-E region for multi-carrier transmissions. Numerical results also highlight the importance of modeling harvester nonlinearity in WIPT system design.
\end{abstract}
