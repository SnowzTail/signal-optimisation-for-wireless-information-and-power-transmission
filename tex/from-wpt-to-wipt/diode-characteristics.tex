Consider the single diode rectifier presented in Figure \ref{fig:single_diode_rectifier} for simplicity. Without loss of generality, the employed diode models hold for general circuits as voltage doubler and bridge rectifiers \cite{Clerckx2017}.

Denote ${v_{{\text{in}}}}(t)$ and ${v_{{\text{out}}}}(t)$ as diode input and output voltages, the voltage across the diode is ${v_{\text{d}}}(t) = {v_{{\text{in}}}}(t) - {v_{{\text{out}}}}(t)$. It determines the current flowing through the diode:

\begin{equation}\label{eqn:diode_characteristics}
  {i_{\text{d}}}(t) = {i_{\text{s}}}\left( {{e^{\frac{{{v_{\text{d}}}(t)}}{{n{v_{\text{t}}}}}}} - 1} \right)
\end{equation}

where ${i_{\text{s}}}$ is the reverse saturation current, $n$ is the ideality factor, and ${{v_{\text{t}}}}$ is the thermal voltage. With a Taylor series expansion around a quiescent point $a = {v_{\text{d}}}(t)$, equation \ref{eqn:diode_characteristics} rewrites as:

\begin{equation}\label{eqn:diode_current_expansion}
  {i_{\text{d}}}(t) = \sum\limits_{i = 0}^\infty  {k_i^\prime } {\left( {{v_{\text{d}}}(t) - a} \right)^i}
\end{equation}

where

\begin{equation}\label{eqn:diode_k_prime}
  k_i^\prime  = \left\{ {
  \begin{array}{*{20}{c}}
    {{i_{\text{s}}}\left( {{e^{\frac{a}{{n{v_{\text{t}}}}}}} - 1} \right),}&{i = 0} \\
    {{i_{\text{s}}}\frac{{{e^{\frac{a}{{n{v_{\text{t}}}}}}}}}{{i!{{\left( {n{v_{\text{t}}}} \right)}^i}}},}&{i \in {\mathbb{N}^ + }}
  \end{array}} \right.
\end{equation}


$k_i^\prime $ relates to the diode parameters and is a constant when $a$ is fixed. Note that the Taylor series expression is a small-signal model that only fits for the nonlinear operation region of the diode. Therefore, equation \ref{eqn:diode_current_expansion} is no longer accurate for a large input voltage ${v_{{\text{in}}}}(t)$, where the diode behavior is dominated by the series resistor and the I-V relationship is linear \cite{Boaventura2013}.

Also, we assume an ideal rectifier with steady-state response that delivers a constant output voltage ${v_{{\text{out}}}}$, whose amplitude is only a function of the peaks of the input voltage ${v_{{\text{in}}}}(t)$ \cite{Curty2005}. Based on the assumptions, a proper choice of voltage drop would be

\begin{equation}\label{eqn:diode_voltage_drop}
  a = \mathbb{E}\left[ {{v_{\text{d}}}(t)} \right] = \mathbb{E}\left[ {{v_{{\text{in}}}}(t) - {v_{{\text{out}}}}} \right] =  - {v_{{\text{out}}}}
\end{equation}

as

\begin{equation}\label{eqn:output_voltage_expectation}
  \mathbb{E}\left[ {{v_{{\text{in}}}}(t)} \right] = \sqrt {{R_{{\text{ant}}}}} \mathbb{E}\left[ {y(t)} \right] = 0
\end{equation}

On top of equation \ref{eqn:diode_voltage_drop} and \ref{eqn:rectifier_input_voltage}, the diode current in \ref{eqn:diode_current_expansion} can be expressed as

\begin{equation}\label{eqn:diode_current}
  {i_{\text{d}}}(t) = \sum\limits_{i = 0}^\infty  {k_i^\prime } {v_{{\text{in}}}}{(t)^i} = \sum\limits_{i = 0}^\infty  {k_i^\prime } R_{{\text{ant}}}^{i/2}y{(t)^i}
\end{equation}

Equation \ref{eqn:diode_current} reveals an explicit relationship between the received waveform $y(t)$ and the diode current ${i_{\text{d}}}(t)$. Nevertheless, as the signal carries both power and information simultaneously, the waveform varies at every symbol period due to the randomness of modulation. Hence, the diode current ${i_{\text{d}}}(t)$ fluctuates with time as well. By taking an expectation over the symbol distribution, the harvested DC current can be modelled as

\begin{equation}\label{eqn:diode_current_expectation}
  {i_{{\text{out}}}} = \mathbb{E}\left[ {{i_{\text{d}}}(t)} \right]
\end{equation}

and the available power is

\begin{equation}\label{eqn:harvested_power}
  P_{{\text{dc}}}^r = i_{{\text{out}}}^2{R_{\text{L}}}
\end{equation}

To investigate the fundamental dependency of harvested power on waveform design, a practical strategy is to approximate equation \ref{eqn:diode_current} with a truncation to the ${n_o}$-th order:

\begin{equation}\label{eqn:output_current_truncation}
  {i_{{\text{out}}}} \approx \sum\limits_{i = 0}^{{n_o}} {k_i^\prime } R_{{\text{ant}}}^{i/2}\mathbb{E}\left[ {y{{(t)}^i}} \right]
\end{equation}

The contribution of odd terms is indeed zero, as $\mathbb{E}\left[ {y{{(t)}^i}} \right] = 0$ for odd $i$. Therefore, we only need to model the even terms:

\begin{equation}\label{eqn:output_current}
  {i_{{\text{out}}}} \approx \sum\limits_{i{\text{ even,i}} \geqslant 0}^{{n_o}} {k_i^\prime } R_{{\text{ant}}}^{i/2}\mathbb{E}\left[ {y{{(t)}^i}} \right]
\end{equation}
