Energy-constrained wireless networks are conventionally powered by batteries. However, limited operation time and high cost in recharging or replacement have become bottlenecks for smart networks as Internet-of-Things (IoT). As a promising solution, Energy Harvesting (EH) from the ambient environment can potentially provide perpetual power to devices. Compared with other renewable resources as solar, wind and water, Radio-Frequency (RF) waves typically contain less energy and are more suitable for low-power applications as Wireless Sensor Network (WSN). With the significant reduction in power requirements of chips and processors, Wireless Power Transfer (WPT) has received recent attentions in both academia and industry \cite{R.Varshney2008,Grover2010,Zhang2013,Hui2014,Krikidis2014,Valenta2014,Boshkovska2015,Ding2015,Costanzo2016,Clerckx2018a}.

On the other hand, RF radiation has been a medium for Wireless Information Transfer (WIT) for more than a century. Naturally, a unified design of Wireless Information and Power transmission (WIPT) is expected to be a prominent solution to power billions of energy-constrained devices while keeping them connected. \cite{R.Varshney2008} first defined a nonlinear concave capacity-energy function and investigated the tradeoff for binary channels and a flat additive white Gaussian noise (AWGN) channel with amplitude-constrained inputs. It was extended to frequency-selective channel in \cite{Grover2010}. Both works were based on the ideal case that information decoding (ID) and EH can be performed individually on the same received signal. In \cite{Zhang2013}, the authors proposed two practical co-located receiver designs named \textit{time switching} (PS) that switches between ID and EH and \textit{power splitting} (TS) that splits the received power into two separate streams. It was demonstrated in \cite{Zhou2013a} that TS can guarantee the same rate as conventional Time-Division Multiple Access (TDMA) while providing reasonable energy. In comparison, PS may lead to higher rate when the power requirement is sufficiently high. A further research \cite{Liu2013} enabled dynamic power splitting that adjusts the power split ratio based on the channel state information (CSI), and proposed a suboptimal low-complexity \textit{antenna switching} scheme. However, the literatures above are mostly based on an oversimplified linear harvester model. To accurately characterize the behavior of the rectenna, \cite{Clerckx2016} derived a tractable nonlinear model and performed an adaptive multisine waveform design accordingly. Realistic simulations showed significant gains in harvested power and stressed the importance of modelling rectifier nonlinearity in wireless system design. The work was extended to multi-input single-output (MISO) WIPT in \cite{Clerckx2018} where a superposition of multicarrier modulated and unmodulated waveform was optimized as a function of CSI with fixed transmit power constraint. It suggests the rectifier nonlinearity can produce a larger rate-energy (R-E) region and favours a different waveform, modulation and input distribution. 